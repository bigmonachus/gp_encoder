%*******************************************************
% Abstract
%*******************************************************
%\renewcommand{\abstractname}{Abstract}
\pdfbookmark[1]{Resumen}{Resumen}
\begingroup
\let\clearpage\relax
\let\cleardoublepage\relax
\let\cleardoublepage\relax

\chapter*{Resumen}

En esta tesis se describe el diseño e implementación de un codificador JPEG que
utiliza programación genética para buscar una mejor compresión. Para esto se
implementa el algoritmo de codificación JPEG, que luego se transforma en una
función de selección. Se implementan un versión secuencial y una paralela, así
como una implementación para el GPU.

El resultado es un nuevo codificador, que escribe archivos en formato JPEG pero
que puede generar imágenes con mejor calidad y menor tamaño.



\vfill

\endgroup

%%% Local Variables:
%%% mode: latex
%%% ispell-local-dictionary: "espanol"
%%% TeX-engine: xetex
%%% TeX-master: "../Tesis_RGC"
%%% End:
