
\newacronym{GPU}{GPU}{Graphics Processing Unit}

\newglossaryentry{compresión de datos}{name=compresión de Datos, description={el acto de transformar la respresentación de la información con el propósito de reducir su tamaño}}

\newglossaryentry{cómputo evolutivo}{name={cómputo evolutivo}, description={una rama de la Inteligencia Artificial que se define por los tipos de algoritmos en los que se enfoca. Son algoritmos que utilizan el concepto de la selección natural para resolver problemas de cómputo. Son problemas de optimización en el sentido matemático.  Es decir, se concentran en encontrar un máximo local o un mínimo local utilizando heurísticas.}}

\newglossaryentry{JPEG}{name={JPEG}, description={Joing Photographic Experts Group}}

\newglossaryentry{teoría de codificación}{name={teoría de codificación}, description={una sub-rama de la Teoría de la Información que
marcó el inicio del estudio formal de la compresión de datos}}

\newglossaryentry{algoritmo evolutivo}{name={algoritmo evolutivo}, description={ver Cómputo Evolutivo}}

\newglossaryentry{BPG}{name={BPG}, description={Better Portable Graphics. Un nuevo formato de compresión para la Web. Por Francis Bellard.}}

\newglossaryentry{compresión sin pérdida}{name={compresión sin pérdida}, description={la compresión que reduce el tamaño
de la representación pero deja la información intacta}}
\newglossaryentry{compresión con pérdida}{name={compresión con pérdida}, description={una transformación que reduce el tamaño de la
representación y que permite un grado de pérdida de información}}

\newglossaryentry{tablas de cuantificación}{name={tablas de cuantificación}, description={matrices de $8\times8$ que determinan en gran parte la calidad de la compresión}}

\newglossaryentry{codificación aritmética}{name={codificación aritmética}, description={Un método de codificación de entropía más sofisticado que los Códigos Huffman}}

\newglossaryentry{códigos de Huffman}{name={códigos de Huffman}, description={Códigos de prefijos que minimizan la longitud de los códigos individuales}}

\newglossaryentry{baseline}{name={baseline}, description={El tipo de compresión JPEG en el que se enfoca este trabajo}}
\newglossaryentry{YUV}{name={YUV}, description={Un espacio de color que separa luminancia y crominancia}}

\newglossaryentry{luminancia}{name={luminancia}, description={La intensidad de la imagen}}

\newglossaryentry{crominancia}{name={crominancia}, description={El color de la imagen}}

\newglossaryentry{VLI}{name={VLI}, description={Entero de longitud variable}}

\newglossaryentry{codificación delta}{name={codificación delta}, description={Se guarda la diferencia entre valores consecutivos en lugar de los valores en si}}

\newglossaryentry{coeficiente DC}{name={coeficiente DC}, description={El primer coeficiente de cada bloque}}

\newglossaryentry{coeficientes AC}{name={coeficientes AC}, description={Los 63 coeficientes después del coeficiente DC}}

\newglossaryentry{tabla unitaria}{name={tabla unitaria}, description={la tabla de cuantificación cuyos coeficientes son 1}}
\newglossaryentry{GPGPU}{name={GPGPU}, description={La práctica de escribir programas en un lenguaje de propósito general para ser ejecutados en un GPU}}

\newglossaryentry{debugging}{name={debugging}, description={También llamado \emph{depurar}, es el proceso de encontrar y arreglar defectos de software}}
\newglossaryentry{profiling}{name={profiling}, description={El proceso de hacer análisis de desempeño de una pieza de software y modificarla para mejorarlo}}
\newglossaryentry{artefactos}{name={artefactos}, description={errores percibibles en la imagen}}
\newglossaryentry{MCU}{name={MCU}, description={Minimum Coded Unit}}

\newglossaryentry{kernel}{name={kernel}, description={función escrita en OpenCL que describe lo que hace un componente de un wavefront}}

\newglossaryentry{wavefront}{name={wavefront}, description={El componente de la aquitectura GPU que ejectura instrucciones vectoriales}}

\newglossaryentry{DCT}{name={DCT}, description={Discrete Cosine Transform}}

\newglossaryentry{}{name={}, description={}}
