\chapter{Introducción}\label{ch:introduction}


\section{Motivación}

% background
% ==== una pagina
La compresión de datos, en pocas palabras, es tomar información y encontrar representaciones que tomen menos espacio. Para lograr esto, se encuentran partes redundantes que puedan abreviars y se remueven partes subjetivamente innecesarias.

Un ejemplo antiguo del uso de redundancia para comprimir información es la invención de la multiplicación para escribir sumas de una manera más corta. e.g. $ a + a + a + a + a $ se comprime con $ 5 * a $ De la misma manera, la exponenciación comprime la multiplicación.

La Teoría de Codificación es una sub-rama de la Teoría de la Información que marcó el inicio del estudio formal de la compresión de datos. Se considera que el inicio de la Teoría de la Información fue la publicación del artículo de Claude E. Shannon ``A Mathematical Theory of Communication'' \citep{shannon}, en el que se introdujo los conceptos de \emph{densidad de información}, \emph{entropía de información} y \emph{redundancia de información}. A principios de la década de 1950, cuando la Teoría de la Codificación estaba naciente, David Huffman inventó lo que hoy conocemos como Codificación Huffman \citep{Huffman}. En 1974, Nasir Ahmed publicó un artículo describiendo su "Transformada de Coseno Discreta" \emph{(DCT)}. Estos conceptos son la base de la compresión JPEG.

A principios de la década de 1990, cuando la popularidad de Internet estaba iniciando su crecimiento exponencial, un comité de matemáticos e ingenieros, llamado "Joint Photographic Experts Group", o JPEG estandarizó un formato de compresión de imágenes \citep{JPEGSTD}. El formato JPEG, soportando varios métodos de compresión, fue adaptado rápidamente y hoy en día es soportado por prácticamente todo programa que soporte imágenes. Imágenes JPEG están incluidas en millones de páginas web y virtualmente todo navegador lo soporta. Sus propiedades lo hacen ideal para contenido fotográfico. Desde que ha habido fotografía digital a nivel consumidor, su formato de elección ha sido JPEG. Los creadores de este formato se basaron en fuertes principios matemáticos, pero su éxito no vino solamente de méritos teóricos; la implementación de referencia fue desarrollada en tándem con la especificación. Esta simbiosis entre ingeniería y teoría es una de las razones por las que JPEG es interesante de estudiar, y una de las razones por las que es un formato ubicuo en la era digital.

Por su naturaleza, los formatos de datos son esclavos a la inercia. Mientras nuestro nuevo conocimiento matemático nos da herramientas que podemos utilizar para crear técnicas más eficientes y compresión de mayor calidad, los datos viejos siguen estando en sus viejos formatos y migrar puede ser difícil o imposible. En el caso de compresión con pérdida, migrar datos a nuevos formatos hace que se pierda parte de la información. Ésta pérdida, en muchos casos, es un precio demasiado alto que pagar para obtener las ventajas de nuevos y mejores formatos con compresión de datos.

Como resultado, se puede notar que la adopción de nuevas técnicas de compresión tiende a ir de la mano con la migración a nuevos medios tecnológicos, e.g. VHS a DVD. Es poco probable que en el corto o mediano plazo veamos una migración tecnológica que nos haga cambiar fundamentalmente la manera que guardamos y compartimos fotografías. JPEG es la técnica de compresión con pérdida es más utilizada en el mundo y no hay razón para pensar que eso va a cambiar en el futuro concebible.

\section{Objetivo}

La compresión JPEG, que será descrita más adelante (TODO anchor), tiene un componente importante para el cual no existe solución óptima. Este componente es un par de matrices de 64 elementos que afecta vitalmente la calidad y la compresión de la imágen.

La especificación incluye matrices de ejemplo, pero la elección de qué par de matrices usar está en las manos de cada codificador JPEG. Estos pares, en la mayoría de los casos, se generan una sola vez y el codificador las usa para siempre. Se encuentran con heurísticas, haciendo pruebas con grandes cantidades de imágenes para encontrar una matriz que sirva bien para la imágen promedio.

El enfoque de este trabajo es producir un codificador JPEG que encuentre un buen par de matrices, de manera individual, para la imagen que se quiere comprimir. Se utiliza un algoritmo genético que evoluciona matrices. La aptitud de cada par de matrices se determina haciendo comparaciones entre la imagen resultante y la original. (TODO mejorar descripcion cuando exista un algoritmo genético)


% [x] los formatos de compresión, por que es interesante para un programador
% [x] factores externos que solidifican formatos. inercia: soporte externo. como esto impide el progreso
% [] descripcion breve de jpeg. hacer nota de que hay una parte que se presta a heurísticas
% [] explicar que jpeg se creó en un mundo distinto. hoy en día tiene sentido gastar los recursos que tenemos en estas heuristicas


