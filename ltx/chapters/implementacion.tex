\chapter{Imlementación JPEG}\label{ch:implementacion}

Se implementó un codificador JPEG, lanzado al dominio público en github bajo el
nombre de TinyJPEG \cite{tiny_jpeg}. La ubicación permanente de esta biblioteca
está en \begin{alltt}https://github.com/serge-rgb/TinyJPEG \end{alltt}

En paralelo con TinyJPEG, se desarrolló un proyecto que extiende TinyJPEG para
incluír el algoritmo evolutivo \cite{gp_encoder}, localizado en
\begin{alltt}https://github.com/serge-rgb/gp_encoder\end{alltt} Entre otras
cosas, se desarrolló una implementación paralela de JPEG que utiliza el GPU
para calcular \emph{DCT}, con el objetivo de hacer una comparación de
desempeño.

Se van a usar los términos \emph{debugging} y \emph{profiling} por su ubiquidad en la literatura. \emph{debugging} se traduce como depurar y se define como el proceso de encontrar y arreglar defectos de código. \emph{profiling} es el proceso de encontrar los puntos en los que un programa puede ser modificado para mejorar el desempeño.

% ============================================================
\section{Lenguajes}
% ============================================================

La elección de lenguaje para TinyJPEG fue C, en particular C99 \cite{c99}.
TinyJPEG fue lanzado con el objetivo de ser una biblioteca reutilizable por
otras personas, y ha encontrado cierto grado de éxito gracias en parte a
integrar una biblioteca de un sólo archivo escrita en C es muy simple para casi
cualquier proyecto. C es un lenguaje ideal para escribir cosas como
codificadores. El lenguaje permite mantener el nivel bajo de abstracción que se
necesita y las herramientas de \emph{debugging} son mejores para C y C++ que
para casi cualquier otro lenguaje.

Escogí CUDA para la implementación de GPGPU. La máquina en la que se desarrolló
tiene una tarjete de video Nvidia, y el soporte de OpenCL en Nvidia deja mucho
que desear en el tiempo en el que se está escribiendo esto. Nvidia proporciona
herramientas excelentes para hacer \emph{debugging} y para \emph{profiling},
que es de particular importancia para un trabajo en el cual el análisis de
desempeño es una parte muy importante.

% ============================================================
\section{Arquitectura}
% ============================================================

% ============================================================
\section{Algoritmo DCT}
% ============================================================


\verb+tiny_jpeg+ contiene dos implementaciones de \emph{DCT}, la que se deriva directamente de la ecuación \ref{eq:dct} y que se describe en \ref{alg:dct}
muestra arriba, y una más rápida, desarrollada por \cite{ahmed_dct}.

Los algoritmos rápidos para calcular la \emph{DCT} están basados en la
observación de que la ecuación \ref{eq:dct} es lineal, y por lo tanto el
cálculo \emph{DCT} se puede ver como una multiplicación de dos matrices de
$8\times8$. La matriz de la derecha es el bloque con los valores de la imagen y
la matriz de la izquierda es la representación como matriz de la ecuación
\ref{eq:dct}.

% ============================================================
\section{Manejo de memoria}
% ============================================================

% ============================================================
\section{Optimización}
% ============================================================

% ============================================================
\section{GPGPU}
% ============================================================
